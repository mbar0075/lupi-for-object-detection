\chapter*{Abstract}
% \addcontentsline{toc}{chapter}{Abstract}

Object detection is widely recognised as a foundational task within computer vision, with applications spanning automation, medical imaging, and surveillance. Although numerous models and methods have been developed, attaining high detection accuracy often requires the utilisation of complex model architectures, especially those based on transformers. These models typically demand extensive computational resources for inference and large-scale annotated datasets for training, both of which contribute to the overall difficulty of the task.

To address these challenges, this work introduces a novel methodology incorporating the Learning Using Privileged Information (LUPI) paradigm within the object detection domain. The proposed approach is compatible with any object detection architecture and operates by introducing privileged information to a teacher model during training. This information is then distilled into a student model, resulting in more robust learning and improved generalisation without increasing the number of model parameters and complexity.

The methodology is evaluated on general-purpose object detection tasks and a focused case study involving litter detection in visually complex, highly variable outdoor environments. These scenarios are especially challenging due to the target objects' small size and inconsistent appearance. Evaluation is conducted both within individual datasets and across multiple datasets to assess consistency and generalisation. A total of 120 models are trained, covering five well-established object detection architectures. Four datasets are used in the evaluation: three focused on UAV-based litter detection and one drawn from the Pascal VOC 2012 benchmark to assess performance in multi-label detection and generalisation.

Experimental results consistently demonstrate improvements in detection accuracy across all model types and dataset conditions when employing the LUPI framework. Notably, the approach yields increases of 0.02 to 0.15 in the strict mean Average Precision (mAP)@50--95 metric, highlighting its robustness across both general-purpose and domain-specific tasks. In nearly all cases, these performance boosts are achieved without increasing the number of parameters or altering the model architecture, confirming the viability of the proposed methodology as a lightweight and effective modification to existing object detection systems.


% 120 trained models for evaluation